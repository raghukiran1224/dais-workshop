\section{Discussion}
The proposed distribution method for DNN inference poses a number of challenges. A crucial consideration is the balance between computation and data transmission. Military coalition operations will likely be limited in both respects, but the optimal balance for deciding what to compute locally versus what to send to other nodes will vary over the coalition network and depend on the following factors:

\begin{itemize}
    \item Local compute availability: the computational resources, and how available those resources are given any other local computing tasks;
    \item Local energy availability: what level of computation and data transmission the device's energy supply supports, and how long this will last given the projected device usage; and
    \item Network transmission capability: the speed and reliability of the network to transmit data between devices.
\end{itemize}

As discussed above, CDLNs can reduce both local computation and local energy consumption over conventional DNNs. The proposed distribution method can further reduce local computation by offloading the processing of higher CDLN layers either to the cloud or to other edge devices. However, this requires the device to transmit data over a potentially slow and unreliable network. A balance must be reached between the cost (both in terms of energy and computation) of running the network locally, and the cost of sending data to other nodes.

A naive implementation of the distributed evaluation method might transmit the full output of a CDLN layer, which for large networks dealing with high dimensional inputs could amount to several hundred kilobytes or more. This could prove unfeasible in a coalition network. A simple approach for reducing data transfer in conventional (non-spiking) CDLN implementations is simply to reduce the bit-depth of the network. This approach has been found to maintain accuracy with networks implemented using 8-bit integers, immediately quartering the required data transfer compared with standard floating point representations~\cite{jacob2018quantization}. Further data reduction methods could also be employed, such as only transmitting layer outputs larger than a certain threshold. However, as discussed in section \ref{sec:spiking}, an even more efficient method would be to implement the CDLN as an SNN and transmit binary spikes between nodes. This not only reduces data transmission requirements, but also improves the robustness of the transmission to unreliable networks: SNNs are inherently robust to stochastic variation in spike trains, meaning that they would cope well with noisy data transmission between nodes.

Another limitation of the proposed method is that it requires copies of the CDLN to be available either in the cloud or on other edge devices. This is unlikely to be the case when the coalition is rapidly formed and dynamic. This limitation may be addressed by researching how to make different DNNs interpretable to each other, such that the output of a layer in one network could be interpreted appropriately by a layer in a different network. In this way, layer outputs from a CDLN could be processed by CDLNs on other devices provided that these CDLNs were \textit{similar enough} to the originating CDLN. The concept of inter-model interpretability has previously been formalized by Dhurandar et al. under the framework of \(\delta\)-interpretability \cite{deltaInterpretability,TIPInterpretability}, but research remains to be done on how to achieve it in practice.

\section{Conlusions}
\label{sec:conclusions}
